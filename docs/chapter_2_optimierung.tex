%!TEX root = ../main.tex
\section{Optimierung}
Nachdem sowohl alle Ideen im finalen Design-Ansatz gesammelt sind sowie die finale Code-Basis einen produktiven Zustand erreicht haben, kann mit der Optimierung des Projektes begonnen werden. Für eine produktive Webseite ist es ausgesprochen wichtig, möglichst wenig Daten bei bestmöglicher Qualität an den Endnutzer herauszugeben. Zwar wurde bei der Entwicklung bereits auf einen sauberen Programmierstil geachtet, dennoch gibt es viele weitere Ansatzpunkte für Verbesserungen.

\subsection{Minifer}
Bilder können in den meisten Fällen extra für das Web optimiert werden. In diesem Projekt ist diesbezüglich ein Gulp.js Plugin zum Einsatz gekommen, welches automatisch eine Komprimierung von Bildern für das WWW (World Wide Web) übernimmt. Genaueres zu Gulp.js folgt im Kapitel \ref{chapter:prepros} auf Seite \pageref{chapter:prepros}.
Neben Bildern gibt es auch im Quellcode viele für den Browser unnötige Bits, deren Übertragung leicht vermieden werden kann. Hierzu zählen vor allem Kommentare, Leerzeilen und Zeilenumbrüche. Da diese allerdings eine große Bedeutung für die Entwickler haben, kann nicht einfach darauf verzichtet werden. Ein entsprechendes Programm (ebenfalls unter Gulp.js) kümmert sich um eine entsprechende Vereinfachung der CSS-Dateien für Webbrowser und verfügt zusätzlich noch über eine Funktion zum Zusammenlegen von gleichen CSS3-Mediaqueries. Im Endeffekt kann somit bei einigen Dateien, ohne zip-Komprimierung, eine Verkleinerung von über 500\% erreicht werden. (Quelltext im Anhang ist zur besseren Lesbarkeit unkomprimiert).

\subsection{Canvas-Element}
Das Canvas-Element hat auf der Startseite für größere FPS-Einbrüche der gesamten Seite gesorgt, weshalb diese neben Optimierungen auf eine 404-Fehler-Seite ausgelagert ist. \ref{descr_canvas}
