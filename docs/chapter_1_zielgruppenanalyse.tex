%!TEX root = ../main.tex
\section{Zielgruppenanalyse}
Es stellt sich vor der tatsächlichen Umsetzung eines Design noch die grundlegende Frage nach der Zielgruppe. Zuerst müssen diesbezüglich einige Fragen betrachet, analysiert und letzlich beantwortet werden, befor mit dem eigentlichen Designprozess begonnen werden kann. Die folgenden Fragen dienen diesem Zweck und dienen als Leitfaden durch den Analyse-Prozess.
\begin{itemize}
	\item Welche Nutzergruppen gibt es und welche Ziele verfolgen sie?
	\item Wie gestaltet sich der Zugriff eines Nutzers auf die Webseite?
	\item Gibt es unter den Nutzern vermehrt beeinträchtigungen, die berücksichtigt werden müssen?
	\item Welches durchschnittliche Know-How kann vom Nutzer erwartet werden? \ldots
\end{itemize}
\subsection{Die Nutzergruppen}
Das Ziel von DigitalHome liegt vor allem darin, Privatpersonen mit modernster Haustechnik zu versorgen. Daher liegen diese vor allem im Fokus dieser Betrachtungen. Hierbei ist zu unterscheiden zwischen denjenigen, die gerade neu bauen und jene die ihr bestehendes Eigenheim aufrüsten möchten.
Momentan leben ca. 80,2 Mio Menschen in Deutschland. Davon befinden sich ca. 22,8 Mio bzw. 28,5\% der Menschen im Alter von 30 - 49 Jahren.\footcite[vgl.][]{zensus2011:alter} Laut einer Studie der LBS liegt das durchschnittlische Alter der privaten Bauherren bei 38 Jahren beim Bau des ersten Eigenheims und 48 Jahren beim Bau eines zweiten.\footcite[vgl.][]{lbs}

In Deutschland werden pro Jahr ca. 250.000 Gebäude gebaut. - Zensus


Daneben gibt es auch sog. Bastler, die sich lieber die einzelnen Komponenten erwerben und diese selbst in ihr eigenheim integrieren und konfigurieren. Auch wenn zu dieser Gruppe für eine genaue Betrachtung nicht die nötigen Daten vorliegen, so ist davon auszugehen, dass für sie höchstwahrscheinlich unser Service den größten Fokus erhält.