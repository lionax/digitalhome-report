%!TEX root = ../main.tex
\section{Zielgruppenanalyse}
Es stellt sich vor der tatsächlichen Umsetzung eines Design noch die grundlegende Frage nach der Zielgruppe. Zuerst müssen diesbezüglich einige Fragen betrachet, analysiert und letzlich beantwortet werden, bevor mit dem eigentlichen Designprozess begonnen werden kann. Dabei wird zwischen zwei Fragetypen unterschieden. Zum einen solche Fragen, welche vorab geklärt werden, um mit dem Designprozess initial in die richtige Richtung zu starten. Sowie Fragen, die während oder nach Abschluss des Designprozesses gestellt werden, um das bestehende Nutzererlebnis weiter zu verbessern.
Zu den Initialen Fragen zählen: 
\begin{itemize}
	\item Welche Nutzergruppen gibt es und welche Ziele verfolgen sie?
	\item Welche Intention leitet den Nutzer auf der Webseite?
	\item Welches durchschnittliche Know-How kann vom Nutzer erwartet werden? 
	\item Wie werden Nutzer die Seite vermutlich aufrufen? 
	\item Welcher Geschmack entspricht dem durchschnittlichen Nutzer? \ldots
\end{itemize}
\subsection{Zahlen und Fakten}
Momentan leben ca. 80,2 Mio Menschen in Deutschland. Davon befinden sich ca. 22,8 Mio bzw. 28,5\% der Menschen im Alter von 30 bis 49 Jahren.\footcite[vgl.][]{zensus2011:alter} Auch laut einer Studie der LBS liegt das durchschnittliche Alter der privaten Bauherren bei 38 Jahren beim Bau des ersten Eigenheims und 48 Jahren beim Bau eines zweiten.\footcite[vgl.][]{lbs} Diesbezüglich ist allerdings anzumerken, dass lediglich 2000 Bauherren befragt wurden und somit die Aussagekraft dieser Studie begrenzt ist. 
Dieser Altersabschnitt lässt zuerst vermuten, dass nicht im allgemeinen von großer Erfahrung der Anwender ausgegangen werden darf, wie beispielsweise bei jüngeren Generationen. Allerdings stellt sich dieser Gedanke als Irrtum heraus, wenn man die Anteile der Internetnutzer im ersten Quartal 2014 betrachtet. Beginnend bei den 10 Jährigen fallen diese Anteile bis zu den 65 Jährigen lediglich auf 50\% bei Männern bzw. 30\% bei Frauen.\footcite[vgl.][]{stabu} Das wiederum gibt grund zur Annahme, 

\subsection{Intention des Nutzers}
Wer diese Webseite aufruft, möchte sich informieren über die Möglichkeiten entweder sein bestehendes Haus mit Smart-Home-Funkionalitäten aufzurüsten oder diese von grundauf mit in den Bau einzubeziehen. Diese Informationen steht also vor allem auf der Startseite im Vordergrund. Generell wird ein Bauherr, der durch den Stress des Bauens vermutlich schon wenig Aufmerksamkeit aufbringen möchte, eine aufgeräumte und informative Präsentation der Inhalte erwarten.

Daneben gibt es auch sog. Bastler, die sich lieber die einzelnen Komponenten erwerben und diese selbst in ihr Eigenheim integrieren und konfigurieren. Zwar steht auch hier die Informationen im Vordergrund, jedoch interessieren sich diese Personen vermutlich eher für spezifischere Informationen, Anleitungen und Service-Produkte.

Im Jahre 2014 waren knapp 50\% der mobilen Internet-Nutzer in Deutschland aus der Altersgruppe 30 bis 49 Jahre.\footcite[vgl.][]{statista:alter} Sicherlich ist einzuwenden, dass solch wichtige Kaufentscheidung wahrscheinlich nicht kurzerhand unterwegs getroffen werden. Dennoch besteht die Möglichkeit, dass vor allem Informationen über mobile Geräte auch vom Sofa aus herangezogen werden. Eine mobile Optimierung der Webseite ist demzufolge als logische Schlussfolgerung nicht zu vernachlässigen. 

Laut einer Statistik von statista.com ist es höchstwahrscheinlich, dass von einem der drei Browser Chrome, Firefox oder Internet Explorer auf die Seite zugegriffen wird. 

\subsection{Progressive Fragen}
Folgende Fragen sollten während der Umsetzung bzw. im Live-Betrieb der Webseite gestellt und analysiert werden, um genauere Antworten zu erhalten. Sie dienen der weiteren Verfeinerung und werden hier nicht weiter erläutert.
\begin{itemize}
	\item Womit wird das Kaufverhalten der Nutzer positiv beeinflusst? 
	\item Wie lange bleibt der Nutzer auf der Seite? 
	\item Wie oft wird ein Kauf tatsächlich abgeschlossen? \ldots
\end{itemize}
