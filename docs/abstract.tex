\begin{abstract} 
	Im Rahmen der Hausarbeit zur Vorlesung Mediendesign an der Hochschule Fulda wird in dieser Arbeit der designtechnische Enwicklungsprozess eines Webshops anhand eines praktischen Beispiels nachvollzogen.

	Zu Beginn werden im Zuge der Vorbereitung eine Shop-Idee ausgearbeitet als auch eine Reihe von Analysen durchgeführt. 
	Im Anschluss steht der eigentliche Design-Prozess im Fokus. Hierbei werden drei grundlegende Designansätze ausgearbeitet, welche im weiteren Verlauf getrennt betrachtet werden. Zu gegebenem Zeitpunkt wird der gelungendste Ansatz ausgewählt und weiter verfeinert. In diesem Zuge werden Unterseiten abgeleitet und optimierungen durchgeführt.
	Ist das Design schlussendlich Finalisiert wird genauer auf die verwendeten Technologien bei der Implementierung eingegangen.
	Letztlich mündet diese Arbeit in einer selbstkritischen Darstellung des Webshops auf Design- und Implementationsebene.
	
\end{abstract} 