%!TEX root = ../main.tex
\section{Grundlegende Designansätze}
Im Zuge der Aufgabenstellung sind drei grundverschiedene Designansätze entstanden, die im Folgenden diskutiert werden. Es sollte in jedem Ansatz eine möglichst differenzierte Idee umgesetzt werden, um eine möglichst heterogene Menge an Ideen zu erhalten.
Einerseits ist ein Ansatz entstanden, der die aktuellen Standards des Genres einzufangen versucht, um dem Nutzer eine möglichst bekannte Nutzererfahrung zu bieten.
Daneben ist auch eine minimalistische Herangehensweise gewählt worden, um ein Design zu erstellen, dass sich explizit auf das wesentliche konzentriert, und alle weiteren Nebensächlichkeiten einfach übergeht. Dem Nutzer soll hierbei die Interaktion völlig ohne Ablenkungen ermöglicht werden.
Letzlich wird auch ein innovativer Ansatz verfolgt, der zum Ziel hat, neue Technologien und Herangehensweisen im Webdesign auszutesten. Dem Nutzer soll hier eine möglichst neue Erfahrung geboten werden, wenngleich auch hier auf bekannte und erwiesenermaßen funktionierende Prinzipien zurückgegriffen wird.
\subsection{Der Design-Prozess}
Bei jedem Ansatz wird der gleiche Prozess verwendet welcher hier kurz erläutert wird. Dieser Prozess ist direkt aus der Lehrveranstaltung abgeleitet.
Beginnend mit dem Comprehensive Dummy wird zuerst die Struktur der Seite definiert. Im Anschluss wird über die Entwicklung in HTML und CSS diskutiert. Es folgt die Definition der Farbgebung und alternativer Farbschemata. Weiterführend wird die typographische Gestaltung betrachtet, Fonts werden ausgewählt sowie über Textkörper und Zeilenabstände nachgedacht. Zum Abschluss wird jeweils eine Strukturanalyse der Webseite betrieben.