\section{Analyse der Konkurrenten}
Potentielle Konkurrenten für Digital Home zu finden ist keine leichte Aufgabe, da Digital Home in seiner Idee Vorreiter ist. Da die Telekom \footcite[Website der Telekom:][]{Tkom} mit ihrer SmartHome Technologie ähnlichkeiten zur Digital Home Idee aufweist, folgt im Folgenden eine Analyse der SmartHome Promotionsseite.

Wie bereits erwähnt ist die Seite der Telekom eine Promotionsseite, wie sie oft bei unterschiedlichsten Produkten zu finden ist. Der Kunde wird durch ein großes Banner auf der Startseite festgehalten, die Bilder sollen ihn neugierig machen. Anschließend kommt ein weiteres Banner mit einer Frau links im Bild. Sie soll vor allem Männer ansprechen, gleichzeitig weckt sie durch ihr scheinbar freundliches Menschendasein bei allen Menschen Vertrauen. Ihre Blicke lenken den User auf ihr Smartphone, welches zur Idee \'Smart Home\' passt. Danach wird der User mit seinen Blicken nach rechts gelenkt, wo sich Argumente für die Smart-Home-Technik finden. Dabei wird die Telekom nicht konkret, sondern versucht auf emotionaler Ebene, den Kunden anzusprechen.

Weiter unten kommt eine kurze Übersicht über die Produkte und Leistungen und anschließend ein Abschnitt zu häufig gestellten Fragen und Bewertungen von Magazinen und Websites. Diese Bewertungen durch scheinbar objektive Institutionen sollen vor allem Vertrauen wecken und Kunden gewinnen, da die Meinung von Dritten von den meisten Menschen als objektive Sicht wahrgenommen wird, wodurch sich der Kunde in der rationalen Entscheidung, diese Produkte der Telekom zu erwerben, sicherer fühlt. Der anschließende Teil mit Meinungen von Kunden und der bereits angesprochene FAQ Bereich geben dem Kunden auf emotionaler Ebene das Gefühl, einer von vielen zu werden, die sich mit dem Produkt wohlfühlen. Dadurch wird dem Kunden die Angst vor einer falschen Kaufentscheidung genommen.

Gegen Ende der Seite befindet sich ein großer pinker Kasten, in dem zwei Links für das weitere Vorgehen der Kunden zu finden sind. Dabei wird von zwei Kundentypen ausgegangen. Der erste ist sofort überzeugt, er kann direkt zum Shop. Der zweite hat noch offene Fragen und ist sich unsicher. Für ihn stehen Ansprechpartner in Telekom Shops zur Verfügung. Dadurch wird das Vertrauen weiter gestärkt, da er nun noch mehr Kontakt zu Menschen (Telekom-Mitarbeitern) hat. Eventuell finden sich in jenen Telekom Shops andere Kunden, die das Bedürfniss nach Telekom Smart Home Produkten durch eigene Meinungen weiter stärken.

Letzlich folgt die Seite strikt dem AIDA-Modell\footcite[vgl.][]{AIDA}. AIDA steht für "'Attention, Interest, Desire, Action"'. Die Aufmerksamkeit des Kunden bekommt die Telekom durch den großen Slider im above-the-fold-Bereich. Große Bilder erregen sofort Aufmerksamkeit, da die Motive vom Menschen sofort erkannt werden. Im Gegensatz dazu steht der Text. Hat man die Aufmerksamkeit des Kunden, so öffnet sich ein Zeitfenster, in dem es den Kunden zu überzeugen gilt. Das übernehmen die Beschriftungen der Bilder im Slider. Durch die Auflistung der Leistungen und Produkte soll das Kaufinteresse geweckt werden. Es wird durch die oben angesprochene Vertrauensarbeit auf emotionaler Ebene verstärkt, während der Abschnitt zum Thema "'Smart Home App"' rationale Gründe hervorbringt, das Produkt zu kaufen.

Die Shoplinks, mit denen der Kunde in Aktion treten soll, werden durch den auffällig pinken Kasten zu "'Call-to-Action"'-Links. Ein weiterer solcher Link befindet sich oben rechts im Bild. Diese drei Links sollen verhindern, dass der Kunde aus dem progressiven AIDA Rythmus rauskommt. Denn dies würde bedeuten, dass der Kunde nicht in Aktion tritt, sondern stattdessen auf der Seite verhaart. Ab diesem Moment lässt sich der Kunde nicht mehr nach dem AIDA steuern, da er sich nun frei auf der Seite bewegen kann. Aus marketingtechnischer Sicht ist dieser User als potetieller Kunde verloren.

Die Telekom versucht mit der Seite, sich als ein Human Era Unternehmen \footcite[vgl.][]{humanEra} darzustellen. Durch Meinungen von anderen Kunden und Magazinen steht das Unternehmen nicht mehr als unfehlbare, übergeordnete, kapitalistische Institution, sondern als soziales, menschliches Subjekt da. Davon hängt vor allem die Grundhaltung und Sympathie des Kunden gegenüber der Telekom ab. Außerdem findet sich auf dem Bild mit der Frau der Hinweis, die Telekom arbeite mit namhaften Partnern zusammen, um Smart Home immer besser zu machen. Dieses progressive Denken zeigt das Bild von ständiger Verbesserung, die eine Folge von Fehlern, Umdenken oder Mangelerscheinungen sind. Dieses Bild des Unternehmens, das Fehler macht, diese jedoch verbessert, zeigt die menschlichen Züge der Telekom und ist typisch für Human Era Unternehmen.

