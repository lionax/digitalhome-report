\section{Grundidee}

Das Handelsunternehmen "Digital Home" verkauft, wie der Name es schon andeutet, digitale Haustechnik, die modular aufgebaut ist, und somit jederzeit erweitert werden kann. Dadurch ist auch eine einfache Wartung von einzelnen Komponenten möglich. Diese soll zentralisiert durch modulare Serversysteme verwaltet werden, worüber man die vernetzte Haustechnik steuern kann. Weitere mögliche Servermodule sind Network-Attached-Storage- (NAS-) Systeme sowie Router für die Verbindung der Server ins Internet. Zusätzlich zählen zu den Produkten neben den modularen Serversystemen noch die interne Haustechnik, also die Steuerungselemente von Heizung, Licht, und auch elektrischen oder elektronischen Geräten; Control Panel, als Schnittstelle zwischen Mensch und dem Serversystem; und Sicherheitssysteme zum Abschirmen vor Internetangriffen durch Verwendung von Hardware-Firewalls. Diese werden in Paketen für private Kunden angeboten.  
\\
Einzelhandelskunden können die Produkte direkt erwerben und weitervermarkten. Zudem bietet Digital Home für ihre Kunden Wartungs- und Servicepakete an, wo die Systeme der Kunden erneuert und auch bei Bedarf repariert oder ausgetauscht werden.
\newline
\newline
Da das Unternehmen mit der Zeit gehen will, möchte es einen modernen Webauftritt haben, der sehr ansprechend ist, damit sich werdende Kunden, die vielleicht ein Paket erwerben wollen; oder aber auch Stammkunden, die sich über Neuigkeiten des Unternehmens informieren wollen oder auch Servicepakete in Anspruch nehmen wollen, schnell zu ihrem Ziel auf der Webseite kommen und sich auch während ihres Aufenthalts auf der Internetseite wohlfühlen, und ihn positiv in Erinnerung behalten.
