%!TEX root = ../main.tex
\section{Fusion der Ideen}
Zwar wurde, wie bereits zuvor erwähnt, der innovative Ansatz von der Jury auserkoren, doch enthalten die anderen Entwürfe dennoch Elemente, die in die weitere Entwicklung übernommen werden können. Teile des innovativen Comps, die als „schön“ angesehen wurden, stellten sich später als ziemlich unflexibel und technisch ungünstig dar.
\subsection{Überarbeitung Logo}
\begin{figure} [hp]
\includegraphics[width=\textwidth]{./img/logo3.png}
\caption{Der finale Logo Entwurf}
\label{logo3}
\end{figure}
Aus Rücksicht auf die Wiedererkennbarkeit wurde der Kreis aus dem Digital-Home-Logo entfernt (siehe Abbildung \ref{logo3}), da er dem Logo eine kreisförmige Kontur gibt, wodurch diese als wesentliches Wiedererkennungsmerkmal des Logos aufgenommen wird. Das Widererkennungsmerkmal soll jedoch das „Haus“ sein, weshalb diese Form nicht eingerahmt werden darf.

\subsection{Einführung Metro-Kacheln}
Nicht nur stellten sich die Produktlinks als unflexibel heraus (Siehe Abschnitt \ref{inno_probs}), sie brechen auch das Layout der Website. Während alle anderen Sections schräge Kanten haben, der Content jedoch in rechteckigen Blöcken organisiert ist, sind jene Links schräge Blöcke, was sie wie eigene Sections aussehen lässt. Die Einführung der Metro-Kacheln macht aus den Links Blöcke, die auf Grund ihrer Form korrekterweise einer Section visuell untergeordnet sind, und damit genauso zum Content gehören, wie auch Überschriften und Texte. Zusätzlich kommen die Kacheln in unterschiedlichen Größen, dadurch können Hierarchien zwischen den Links gebildet werden, wodurch man Highlights betonen oder weniger wichtige Links in den Hintergrund rücken kann. Außerdem dürfen die Kacheln einen beliebigen Abstand zum rechten Rand der Section haben, ohne das auffällige Stilbrüche auftreten, wodurch das Kachelsystem nochmals flexibler wird.

\subsection{Menu}
Um schneller navigieren zu können, wurde die Menüleiste mit Symbolen ausgestattet, ohne dass man das Menü öffnen muss und ohne dass der User Text lesen muss. Für eine noch bessere UX (User Experience) auf Desktops muss das Menü nicht mehr durch einen Klick geöffnet werden. Stattdessen ist dass nun optional durch hovern der Symbolleiste zu erreichen. In der Folge kann sich der Nutzer schneller zu anderen Seiten bewegen. Dieses Hover-Verhalten gilt nicht nur für die Symbolleiste, das Menü öffnet sich auch bei Hovern der Suchleiste. Dieses Verhalten macht Sinn, da der User sowieso versucht durch die Webseite zu navigieren um Inhalte zu finden, die er sucht. 
Möglicherweise wurde das Menü auch gar nicht oder nicht als solches wahrgenommen. Dadurch ist das eine gute Gelegenheit, dem Nutzer zu zeigen, wie er navigieren kann. Findet er keinen passenden Link, kann er trotzdem die Suchleiste benutzen.
