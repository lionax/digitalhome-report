\section{Fusion der Ideen}
Zwar wurde der innovative Comp im Plenum als bester Comp gewählt, doch enthalten die anderen Entwürfe Elemente, die übernommen werden sollten. Teile des innovativen Comps, die als „schön“ angesehen wurden, stellten sich später als ziemlich unflexibel und technisch ungünstig dar.
\subsection{Überarbeitung Logo}
\begin{figure} [tp]
\includegraphics[width=\textwidth]{./img/logo3.png}
\caption{Der finale Logo Entwurf}
\label{logo3}
\end{figure}
Aus Rücksicht auf die Wiedererkennbarkeit wurde der Kreis aus dem DigitalHome Logo entfernt (siehe Abbildung \ref{logo3}), da er dem Logo eine kreisförmige Kontur gibt, wodurch diese als wesentliches Wiedererkennungsmerkmal des Logos aufgenommen wird. Das Widererkennungsmerkmal soll jedoch das „Haus“ sein, weshalb diese Form nicht eingerahmt werden darf.

\subsection{Einführung Metro Kacheln}
\subsection{Style Guide}
Nach der Fusion aller Ideen können wir ein einheitliches Branding für DigitalHome festlegen, das wir brauchen, um Entscheidungen zur Gestaltung der Website zu treffen.
\subsection{Menu}
Um schneller navigieren zu können, wurde die Menüleiste mit Symbolen ausgestattet, ohne dass man das Menü öffnen muss und ohne dass der User Text lesen muss. Für eine noch bessere UX auf Desktops muss das Menü nicht mehr durch einen Klick, sondern kann durch hovern der Symbolleiste geöffnet werden. In der Folge kann sich der Nutzer schneller zu anderen Seiten bewegen. Dieses Hover Verhalten gilt nicht nur für die Symbolleiste, das Menü öffnet sich auch bei hovern der Suchleiste. Das macht deshalb Sinn, denn wer etwas auf einer Website sucht, der versucht damit, von der aktuellen Seite auf eine andere zu gelangen, um andere Informationen zu bekommen. Vielleicht wurde das Menü auch gar nicht oder nicht als solches wahrgenommen. Eine gute Gelegenheit, dem Nutzer zu zeigen, wie er navigieren kann. Findet er keinen passenden Link, kann er trotzdem die Suchleiste benutzen.
